\documentclass{article} 
\title{Baroclinic Instabilities} 
\author{Jacob Perez} 
\begin{document} 
\maketitle
\tableofcontents 
\section{Introduction} 

A baroclinic instability is a fluid dynamical instability, that can be used to explain the generation and growth of weather phenomena in the mid-latitudes. The main two sources of instabilities are the vertical and horizontal shear of the mean background wind profile. The requirement for vertical shearing implies that a meridional temperature gradient, which in turn means there is available potential energy (APE) which can be released and transferred to the disturbances in the flow. The process of transferring APE into the disturbances is a baroclinic instability.  
\subsection{Qualitative example} 
A qualitative description of this process is provided by Pedlosky, who uses a motivating example given by Eady, to describe the mechanism behind a baroclinic instability.  In figure 1 below we have the following situation, a constant potential temperature ($\theta_*$) surface tilts upward in the meridional plane by an angle $\alpha$. Consider a fluid parcel, starting at position A be displaced to position B. By considering the change in density when moving from A to B, the restoring force becomes  
\begin{equation}
  E_* = \frac{g}{\theta_*}\frac{\partial\theta_*}{\partial z_*}\sin\phi\left[d_{z_*}-d_{y_*}\left(\frac{\partial z_*}{\partial y_*}\right)_{\theta_*}\right]
\end{equation}
where $d_{y_*}$ and $d_{z_*}$ are displacements in the $y$ and $z$ planes respectively and $\phi=\tan^{-1}(d_{z_*}/d_{y_*})$ is the angle of displacement. From this we can deduce that any vertical displacement ($d_{y_*}=0$ and $\sin\phi = 1$) reduces $E_*$ to the Brunt-V\"as\"al\"a frequency. For positive restoring forces we see that the system will return to an equilibrium state, but for a negative restoring force occuring when the fluid element satisfies
\begin{equation}
  0 < \tan\phi < \left(\frac{\partial z_*}{\partial y_*}\right),
\end{equation}
will cause the buyoancy force to accelerate the fluid parcel further and further away from its initial position. This is the main idea behind a baroclinic instability. For the fluid within this section defined by the angle $\phi$ lower density fluid will rise and the higher density fluid will sink, in turn releasing potential energy. The idea of varying densitys can be directly linked variations in temperature of the fluid, meaning that a baroclinic instability is a form of thermal convection. 

\section{The Charney Model} 
The first publication on theoretical models of baroclinic instabilities was by Jules Charney in 1947. Charneys' model is in fact a slightly more complicated model than a later model proposed by Eady in 1949, with the main difference being the inclusion of the $\beta$ effect in the model proposd by Charney.  


\subsection{Formulation}
\subsection{Results}
\subsection{Drawbacks} 
\section{Applications to Weather Forecasting} 

\section{Summary}

\end{document}
